\documentclass[adobefonts, nocap]{ctexart}
\usepackage{amsmath}
\begin{document}
  \title{计算机系统结构第一次作业}
  \author{李雨田\hspace{1em}2010012193\hspace{1em}计14}
  \maketitle
  \section*{1}
    未改进部分的百分比是$1-f_{1}-f_{2}$,所以加速比是
    \[
      \frac{T_{0}}{T_{n}}=\frac{1}{1-f_{1}-f_{2}+\frac{f_{1}}{S_{1}}+\frac{f_{2}}{S_{2}}}.
    \]
  \section*{2}
    设改进后改进部件执行时间为$x$,则改进后系统执行时间是$2x$.由于改进部件速度提高了$10$倍,所以改进部件改进前执行时间为$10x$,在加上未改进部件的执行时间$x$,总时间是$11x$.所以加速比是
    \[
      \frac{T_{0}}{T_{n}}=\frac{11}{2}.
    \]

    但如果严格按照“提高了$10$倍”的定义来算的话,那么改进部件改进前执行时间为$11x$(而不是之前算出来的$10x$),在加上未改进部件的执行时间$x$,总时间是$12x$.所以加速比是
    \[
      \frac{T_{0}}{T_{n}}=6.
    \]
\end{document}
