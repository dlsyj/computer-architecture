\documentclass[adobefonts, nocap]{ctexart}
\usepackage{amsmath}
\usepackage{amsfonts}
\usepackage{listings}
\usepackage{xcolor}
\usepackage{graphicx}
\usepackage{siunitx}
\usepackage{hyperref}
\hypersetup{
  colorlinks = true,
  linkcolor = blue,
  unicode = true
}
\lstset{
  language = C++,
  basicstyle = \small\ttfamily,
  keywordstyle = \small\ttfamily\color{red},
  stringstyle = \color{gray},
  numbers = left,
  numberstyle = \small,
  numbersep = 5pt,
  frame = leftline,
  showstringspaces = false
}
\def\D{\mathrm{d}}
\begin{document}
  \title{计算机系统结构第三次作业}
  \author{李雨田\hspace{1em}2010012193\hspace{1em}计14}
  \maketitle
  \section*{3.8}
    如图所示,可以先计算$A_{i}\times B_{i}$, $i\in \{1,2,3\}$,在计算$A_{4}\times B_{4}$之前先计算出$A_{1}\times B_{1}+A_{2}\times B_{2}$,然后再算出剩下的值.

    \begin{center}
      \includegraphics[width=10cm]{1-crop.pdf}
    \end{center}

    在$18$个$\Delta t$时间中,给出了$7$个结果,所以吞吐率为
    \[
      TP=\frac{7}{18\Delta t}.
    \]

    如果不适用流水线,产生$7$个结果总共需要时间$(4\times 4+3\times 4)\Delta t=28\Delta t$,所以加速比为
    \[
      S=\frac{28\Delta t}{18\Delta t}=\frac{14}{9}.
    \]

    流水线的效率可由阴影区的面积和总面积的比值求得
    \[
      E=\frac{28}{5\times 18}=\frac{14}{45}.
    \]
  \section*{3.9}
    根据预约表,可以得到禁止表
    \[
      F=\{1,3,4,8\}.
    \]

    写出初始冲突向量
    \[
      C_{0}=(10001101).
    \]

    再根据初始冲突向量可以画出状态转换图.

    \begin{center}
      \includegraphics[width=10cm]{2-crop.pdf}
    \end{center}

    可以看出$(2,5)$是最优的调度策略,平均时间间隔是$3.5\Delta t$,即可得吞吐率
    \[
      TP=\frac{1}{3.5\Delta t}=\frac{2}{7\Delta t}.
    \]

    如果连续输出$6$个任务,分别相隔$2\Delta t, 5\Delta t, 2\Delta t, 5\Delta t, 2\Delta t, 5\Delta t$进入流水线,最后一个任务执行还需要时间$9\Delta t$,总共时间为$30\Delta t$.实际吞吐率为
    \[
      TP=\frac{6}{30\Delta t}=\frac{1}{5\Delta t}.
    \]
    
    实际吞吐率总是小于理论上的最大吞吐率,这个结论得到验证.
  \section*{3.10}
    使用相同的流程,首先根据预约表得到禁止表
    \[
      F=\{1,3,6\}.
    \]

    写出初始冲突向量
    \[
      C_{0}=(100101).
    \]

    再根据初始冲突向量可以画出状态转换图.

    \begin{center}
      \includegraphics[width=10cm]{3-crop.pdf}
    \end{center}

    可以看出允许不等时间间隔调度时,$(2,2,5)$是最优的调度策略,平均时间间隔是$3\Delta t$,可得到吞吐率
    \[
      TF=\frac{1}{3\Delta t}.
    \]

    等时间间隔调度时,最优调度策略是$(5)$,吞吐率
    \[
      TF=\frac{1}{5\Delta t}.
    \]

    连续输入$10$个任务时,采用不等时间间隔调度耗时$(2+2+5+2+2+5+2+2+5+2+7)\Delta t=36\Delta t$,实际吞吐率
    \[
      TF=\frac{10}{36\Delta t}=\frac{5}{18\Delta t},
    \]
    加速比为
    \[
      S=\frac{10\times 7\Delta t}{36\Delta t}=\frac{35}{18}.
    \]

    采用等时间间隔调度则耗时$(10\times 5+7)\Delta t=57\Delta t$,实际吞吐率
    \[
      TF=\frac{10}{57\Delta t},
    \]
    加速比为
    \[
      S=\frac{10\times 7\Delta t}{57\Delta t}=\frac{70}{57}.
    \]
\end{document}
