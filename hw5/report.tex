\documentclass[adobefonts, nocap]{ctexart}
\usepackage{amsmath}
\usepackage{amsfonts}
\usepackage{listings}
\usepackage{xcolor}
\usepackage{graphicx}
\usepackage{siunitx}
\usepackage{hyperref}
\hypersetup{
  colorlinks = true,
  linkcolor = blue,
  unicode = true
}
\lstset{
  language = C,
  basicstyle = \small\ttfamily,
  keywordstyle = \small\ttfamily\color{red},
  stringstyle = \color{gray},
  numbers = left,
  numberstyle = \small,
  numbersep = 5pt,
  frame = leftline,
  showstringspaces = false
}
\def\D{\mathrm{d}}
\begin{document}
  \title{计算机系统结构第五次作业}
  \author{李雨田\hspace{1em}2010012193\hspace{1em}计14}
  \maketitle
  \section*{1}
    \subsection*{(1)}
      带来额外CPI开销的有两种可能.一种是BTB命中但是分支预测错误.另一种是BTB没有命中.

      所以额外开销为
      \[
        90\%\times\left(100\%-90\%\right)\times 4+10\%\times 3=0.66.
      \]

      程序执行的CPI为
      \[
        15\%\times 0.66+1=1.099.
      \]
    \subsection*{(2)}
      采用固定$2$个时钟周期延迟的分支处理,程序执行的CPI为
      \[
        15\%\times 2+1=1.3.
      \]

      可见使用BTB程序执行速度更快.
  \section*{2}
    令无条件分支指令的延迟为$x$,则
    \[
      1+5\%\times x=1.1.
    \]

    得到
    \[
      x=2.
    \]

    允许无条件分支指令进入BTB时,程序执行的CPI为
    \[
      1+5\%times\left(100\%-90\%\right)\times 2=1.01.
    \]
\end{document}
